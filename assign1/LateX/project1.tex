\documentclass[10pt,letterpaper,draftclsnofoot,onecolumn]{IEEEtran}

\usepackage{graphicx}                                        
\usepackage{amssymb}                                         
\usepackage{amsmath}                                         
\usepackage{amsthm}                                          

\usepackage{alltt}                                           
\usepackage{float}
\usepackage{color}
\usepackage{url}

\usepackage{balance}
\usepackage[TABBOTCAP, tight]{subfigure}
\usepackage{enumitem}
%\usepackage{pstricks, pst-node}

\usepackage{geometry}
\geometry{margin=0.75in}

%random comment

\newcommand{\cred}[1]{{\color{red}#1}}
\newcommand{\cblue}[1]{{\color{blue}#1}}

\newcommand{\toc}{\tableofcontents}

\usepackage{hyperref}

\def\name{Will Sims, Aidan O'Malley}

%pull in the necessary preamble matter for pygments output
%\usepackage{fancyvrb}
\usepackage{color}
\usepackage[latin1]{inputenc}


\makeatletter
\def\PY@reset{\let\PY@it=\relax \let\PY@bf=\relax%
    \let\PY@ul=\relax \let\PY@tc=\relax%
    \let\PY@bc=\relax \let\PY@ff=\relax}
\def\PY@tok#1{\csname PY@tok@#1\endcsname}
\def\PY@toks#1+{\ifx\relax#1\empty\else%
    \PY@tok{#1}\expandafter\PY@toks\fi}
\def\PY@do#1{\PY@bc{\PY@tc{\PY@ul{%
    \PY@it{\PY@bf{\PY@ff{#1}}}}}}}
\def\PY#1#2{\PY@reset\PY@toks#1+\relax+\PY@do{#2}}

\expandafter\def\csname PY@tok@gd\endcsname{\def\PY@tc##1{\textcolor[rgb]{0.63,0.00,0.00}{##1}}}
\expandafter\def\csname PY@tok@gu\endcsname{\let\PY@bf=\textbf\def\PY@tc##1{\textcolor[rgb]{0.50,0.00,0.50}{##1}}}
\expandafter\def\csname PY@tok@gt\endcsname{\def\PY@tc##1{\textcolor[rgb]{0.00,0.25,0.82}{##1}}}
\expandafter\def\csname PY@tok@gs\endcsname{\let\PY@bf=\textbf}
\expandafter\def\csname PY@tok@gr\endcsname{\def\PY@tc##1{\textcolor[rgb]{1.00,0.00,0.00}{##1}}}
\expandafter\def\csname PY@tok@cm\endcsname{\let\PY@it=\textit\def\PY@tc##1{\textcolor[rgb]{0.25,0.50,0.50}{##1}}}
\expandafter\def\csname PY@tok@vg\endcsname{\def\PY@tc##1{\textcolor[rgb]{0.10,0.09,0.49}{##1}}}
\expandafter\def\csname PY@tok@m\endcsname{\def\PY@tc##1{\textcolor[rgb]{0.40,0.40,0.40}{##1}}}
\expandafter\def\csname PY@tok@mh\endcsname{\def\PY@tc##1{\textcolor[rgb]{0.40,0.40,0.40}{##1}}}
\expandafter\def\csname PY@tok@go\endcsname{\def\PY@tc##1{\textcolor[rgb]{0.50,0.50,0.50}{##1}}}
\expandafter\def\csname PY@tok@ge\endcsname{\let\PY@it=\textit}
\expandafter\def\csname PY@tok@vc\endcsname{\def\PY@tc##1{\textcolor[rgb]{0.10,0.09,0.49}{##1}}}
\expandafter\def\csname PY@tok@il\endcsname{\def\PY@tc##1{\textcolor[rgb]{0.40,0.40,0.40}{##1}}}
\expandafter\def\csname PY@tok@cs\endcsname{\let\PY@it=\textit\def\PY@tc##1{\textcolor[rgb]{0.25,0.50,0.50}{##1}}}
\expandafter\def\csname PY@tok@cp\endcsname{\def\PY@tc##1{\textcolor[rgb]{0.74,0.48,0.00}{##1}}}
\expandafter\def\csname PY@tok@gi\endcsname{\def\PY@tc##1{\textcolor[rgb]{0.00,0.63,0.00}{##1}}}
\expandafter\def\csname PY@tok@gh\endcsname{\let\PY@bf=\textbf\def\PY@tc##1{\textcolor[rgb]{0.00,0.00,0.50}{##1}}}
\expandafter\def\csname PY@tok@ni\endcsname{\let\PY@bf=\textbf\def\PY@tc##1{\textcolor[rgb]{0.60,0.60,0.60}{##1}}}
\expandafter\def\csname PY@tok@nl\endcsname{\def\PY@tc##1{\textcolor[rgb]{0.63,0.63,0.00}{##1}}}
\expandafter\def\csname PY@tok@nn\endcsname{\let\PY@bf=\textbf\def\PY@tc##1{\textcolor[rgb]{0.00,0.00,1.00}{##1}}}
\expandafter\def\csname PY@tok@no\endcsname{\def\PY@tc##1{\textcolor[rgb]{0.53,0.00,0.00}{##1}}}
\expandafter\def\csname PY@tok@na\endcsname{\def\PY@tc##1{\textcolor[rgb]{0.49,0.56,0.16}{##1}}}
\expandafter\def\csname PY@tok@nb\endcsname{\def\PY@tc##1{\textcolor[rgb]{0.00,0.50,0.00}{##1}}}
\expandafter\def\csname PY@tok@nc\endcsname{\let\PY@bf=\textbf\def\PY@tc##1{\textcolor[rgb]{0.00,0.00,1.00}{##1}}}
\expandafter\def\csname PY@tok@nd\endcsname{\def\PY@tc##1{\textcolor[rgb]{0.67,0.13,1.00}{##1}}}
\expandafter\def\csname PY@tok@ne\endcsname{\let\PY@bf=\textbf\def\PY@tc##1{\textcolor[rgb]{0.82,0.25,0.23}{##1}}}
\expandafter\def\csname PY@tok@nf\endcsname{\def\PY@tc##1{\textcolor[rgb]{0.00,0.00,1.00}{##1}}}
\expandafter\def\csname PY@tok@si\endcsname{\let\PY@bf=\textbf\def\PY@tc##1{\textcolor[rgb]{0.73,0.40,0.53}{##1}}}
\expandafter\def\csname PY@tok@s2\endcsname{\def\PY@tc##1{\textcolor[rgb]{0.73,0.13,0.13}{##1}}}
\expandafter\def\csname PY@tok@vi\endcsname{\def\PY@tc##1{\textcolor[rgb]{0.10,0.09,0.49}{##1}}}
\expandafter\def\csname PY@tok@nt\endcsname{\let\PY@bf=\textbf\def\PY@tc##1{\textcolor[rgb]{0.00,0.50,0.00}{##1}}}
\expandafter\def\csname PY@tok@nv\endcsname{\def\PY@tc##1{\textcolor[rgb]{0.10,0.09,0.49}{##1}}}
\expandafter\def\csname PY@tok@s1\endcsname{\def\PY@tc##1{\textcolor[rgb]{0.73,0.13,0.13}{##1}}}
\expandafter\def\csname PY@tok@sh\endcsname{\def\PY@tc##1{\textcolor[rgb]{0.73,0.13,0.13}{##1}}}
\expandafter\def\csname PY@tok@sc\endcsname{\def\PY@tc##1{\textcolor[rgb]{0.73,0.13,0.13}{##1}}}
\expandafter\def\csname PY@tok@sx\endcsname{\def\PY@tc##1{\textcolor[rgb]{0.00,0.50,0.00}{##1}}}
\expandafter\def\csname PY@tok@bp\endcsname{\def\PY@tc##1{\textcolor[rgb]{0.00,0.50,0.00}{##1}}}
\expandafter\def\csname PY@tok@c1\endcsname{\let\PY@it=\textit\def\PY@tc##1{\textcolor[rgb]{0.25,0.50,0.50}{##1}}}
\expandafter\def\csname PY@tok@kc\endcsname{\let\PY@bf=\textbf\def\PY@tc##1{\textcolor[rgb]{0.00,0.50,0.00}{##1}}}
\expandafter\def\csname PY@tok@c\endcsname{\let\PY@it=\textit\def\PY@tc##1{\textcolor[rgb]{0.25,0.50,0.50}{##1}}}
\expandafter\def\csname PY@tok@mf\endcsname{\def\PY@tc##1{\textcolor[rgb]{0.40,0.40,0.40}{##1}}}
\expandafter\def\csname PY@tok@err\endcsname{\def\PY@bc##1{\setlength{\fboxsep}{0pt}\fcolorbox[rgb]{1.00,0.00,0.00}{1,1,1}{\strut ##1}}}
\expandafter\def\csname PY@tok@kd\endcsname{\let\PY@bf=\textbf\def\PY@tc##1{\textcolor[rgb]{0.00,0.50,0.00}{##1}}}
\expandafter\def\csname PY@tok@ss\endcsname{\def\PY@tc##1{\textcolor[rgb]{0.10,0.09,0.49}{##1}}}
\expandafter\def\csname PY@tok@sr\endcsname{\def\PY@tc##1{\textcolor[rgb]{0.73,0.40,0.53}{##1}}}
\expandafter\def\csname PY@tok@mo\endcsname{\def\PY@tc##1{\textcolor[rgb]{0.40,0.40,0.40}{##1}}}
\expandafter\def\csname PY@tok@kn\endcsname{\let\PY@bf=\textbf\def\PY@tc##1{\textcolor[rgb]{0.00,0.50,0.00}{##1}}}
\expandafter\def\csname PY@tok@mi\endcsname{\def\PY@tc##1{\textcolor[rgb]{0.40,0.40,0.40}{##1}}}
\expandafter\def\csname PY@tok@gp\endcsname{\let\PY@bf=\textbf\def\PY@tc##1{\textcolor[rgb]{0.00,0.00,0.50}{##1}}}
\expandafter\def\csname PY@tok@o\endcsname{\def\PY@tc##1{\textcolor[rgb]{0.40,0.40,0.40}{##1}}}
\expandafter\def\csname PY@tok@kr\endcsname{\let\PY@bf=\textbf\def\PY@tc##1{\textcolor[rgb]{0.00,0.50,0.00}{##1}}}
\expandafter\def\csname PY@tok@s\endcsname{\def\PY@tc##1{\textcolor[rgb]{0.73,0.13,0.13}{##1}}}
\expandafter\def\csname PY@tok@kp\endcsname{\def\PY@tc##1{\textcolor[rgb]{0.00,0.50,0.00}{##1}}}
\expandafter\def\csname PY@tok@w\endcsname{\def\PY@tc##1{\textcolor[rgb]{0.73,0.73,0.73}{##1}}}
\expandafter\def\csname PY@tok@kt\endcsname{\def\PY@tc##1{\textcolor[rgb]{0.69,0.00,0.25}{##1}}}
\expandafter\def\csname PY@tok@ow\endcsname{\let\PY@bf=\textbf\def\PY@tc##1{\textcolor[rgb]{0.67,0.13,1.00}{##1}}}
\expandafter\def\csname PY@tok@sb\endcsname{\def\PY@tc##1{\textcolor[rgb]{0.73,0.13,0.13}{##1}}}
\expandafter\def\csname PY@tok@k\endcsname{\let\PY@bf=\textbf\def\PY@tc##1{\textcolor[rgb]{0.00,0.50,0.00}{##1}}}
\expandafter\def\csname PY@tok@se\endcsname{\let\PY@bf=\textbf\def\PY@tc##1{\textcolor[rgb]{0.73,0.40,0.13}{##1}}}
\expandafter\def\csname PY@tok@sd\endcsname{\let\PY@it=\textit\def\PY@tc##1{\textcolor[rgb]{0.73,0.13,0.13}{##1}}}

\def\PYZbs{\char`\\}
\def\PYZus{\char`\_}
\def\PYZob{\char`\{}
\def\PYZcb{\char`\}}
\def\PYZca{\char`\^}
\def\PYZam{\char`\&}
\def\PYZlt{\char`\<}
\def\PYZgt{\char`\>}
\def\PYZsh{\char`\#}
\def\PYZpc{\char`\%}
\def\PYZdl{\char`\$}
\def\PYZti{\char`\~}
% for compatibility with earlier versions
\def\PYZat{@}
\def\PYZlb{[}
\def\PYZrb{]}
\makeatother


%% The following metadata will show up in the PDF properties
\hypersetup{
   colorlinks = false,
   urlcolor = black,
   pdfauthor = {\name},
   pdfkeywords = {cs444},
   pdftitle = {CS 444 Project 1: Getting Acquainted},
   pdfsubject = {CS 444 Project 1},
   pdfpagemode = UseNone
}

\parindent = 0.0 in
\parskip = 0.1 in

\begin{document}

%input the pygmentized output of mt19937ar.c, using a (hopefully) unique name
%this file only exists at compile time. Feel free to change that.

\begin{titlepage}
\title{CS 444 Project 1: Getting Acquainted}
\author
{\IEEEauthorblockN{Will Sims, Aidan O'Malley\\
}
\IEEEauthorblockA{
CS 444\\
Fall 2017
}}
    \maketitle
    \vspace{2cm}
    \begin{abstract}
        \noindent Abstract goes here
    \end{abstract}

\end{titlepage}

\section{Log of Commands}
\noindent Log of commands used to perform the requested actions
\begin{description}
\item \texttt{cd /scratch/fall2017}
\item \texttt{ls}
\item \texttt{mkdir 19}
\item \texttt{ls}
\item \texttt{git clone git://git.yoctoproject.org/linux-yocto-3.19}
\item \texttt{git status}
\item \texttt{git add .}
\item \texttt{git commit -m "initial commit"}
\item \texttt{git status}
\item \texttt{cd ..}
\item \texttt{cd fall2017}
\item \texttt{cd ..}
\item \texttt{./acl\_open ../fall2017/19 simsw}
\item \texttt{cd ..}
\item \texttt{cd fall2017}
\item \texttt{cd 19}
\item \texttt{cd linux-yocto-3.19/}
\item \texttt{git checkout v3.19.2}
\item \texttt{source ../../../opt/poky/1.8/environment-setup-i586-poky-linux }
\item \texttt{cp ../../../files/bzImage-qemux86.bin .}
\item \texttt{cp ../../../files/core-image-lsb-sdk-qemux86.ext4 .}
\item \texttt{make menuconfig}
\item \texttt{qemu-system-i386 -gdb tcp::5519 -S -nographic -kernel bzImage-qemux86.bin -drive file=core-image-lsb-sdk-qemux86.ext4,if=virtio -enable-kvm -net none -usb -localtime --no-reboot --append "root=/dev/vda rw console=ttyS0 debug"}
\item \texttt{\$GDB}
\item \texttt{target remote :5519}
\item \texttt{qemu-system-i386 -gdb tcp::5519 -S -nographic -kernel arch/x86/boot/bzImage -drive file=core-image-lsb-sdk-qemux86.ext4,if=virtio -enable-kvm -net none -usb -localtime --no-reboot --append "root=/dev/vda rw console=ttyS0 debug"}
\end{description}

\section{Flag Definitions}
\noindent An explanation of each and every flag in the listed qemu command-line is below:
\begin{description}
\item [-gdb tcp::5519] this flag makes it so that we wait for a gdb connection on tcp::5519 before the VM runs.
\item [-S] this flag makes it so that the CPU does not start at startup.
\item [-nographic] this disables qemu's graphical output so that we can use qemu as a simple command line application.
\item [-kernel arch/x86/boot/bzImage] this flag is to setup the kernel and specify the file to use for the kernel image.
\item [-drive file=core-image-lsb-sdk-qemux86.ext4,if=virtio] this flag defines a new drive to use. We set the file and then the interface (in this case virtio) that the drive is connected to.
\item [-enable-kvm] this flag enables full kernel-based virtual machine virtualization support.
\item [-net none] this flag is used to create a new network interface card, when the flag isn’t used a single network interface card is created. So in this case we set the net to none so that no network interface card is created or used.
\item [-usb] this flag enables the usb driver.
\item [-localtime] this flag sets the rtc to local time.
\item [--no-reboot] this flag tells the VM that it should exit instead of reboot.
\item [--append "root=/dev/vda rw console=ttyS0 debug"] the append flag specifies the kernel command line characteristics.
\end{description}

\section{Concurrency Questions}
\noindent\textbf{What do you think the main point of this assignment is?}

\indent{We thought the main point of the assignment was to improve and freshen up our C programming skills after a long summer working in other languages. We also thought the assignment was given to get us students thinking about what some good ways of implementing programs with multiple threads are. The producer-consumer scheme is clearly a fairly simple way of implementing multiple threads in a program so it was a good refresher and it was thought provoking.}

\noindent\textbf{How did you personally approach the problem?}

\indent{We approached the problem by first writing pseudocode using the “Little Book of Semaphores” as reference. We knew we were going to need subroutines for the producer and the consumer. We knew there would have to be a main where most of the setup was going to take place. We added comments in these functions based on the problem statement to get a skeleton of our program, for example, in the consumer function we made comments specifying the flow of the subroutine such as “//give consumer exclusive access to buffer”, “//consume buffer item”, “//unlock mutex during sleep”. We created the struct and the buffer immediately as well so that we had a good idea of the data we’d be working with. From their we implemented the random number generation. We used code provided in the class materials to check if rdrand could be used on the system running the program. Based on how this function completed, a function pointer is set to either use the rdrand number generator or the mt19937 number generator. This function pointer is then used in the main logic of the program. After getting this set up we moved into setting up the threads, the cond threads, and the mutex to create the producer consumer relationship, this setup was mostly done in the main as well as initializing some global variables. Lastly we implemented the functionality of the producer and consumer functions using the comments we made earlier as well as “The Little Book of Semaphores” as reference. We decided a good design would be to keep a global index of where we are in the buffer, we then use that index to be able to add to the buffer with the producer. We decided that the consumer would just pop the first element in the buffer off when it consumed an item and we’d decrement the shared index variable. From there we were able to complete the concurrency assignment.}

\noindent\textbf{How did you ensure your solution was correct?}

\indent{We ensured that the program was correct by testing against the problem statement requirements. We ensured that the consumer wouldn’t try to consume anything from the buffer if the buffer was empty. We did this by leaving the buffer empty and seeing if the consumer thread would block correctly. A similar check was done for the case that the buffer is full and the producer is told to produce another item, in this case the producer blocks correctly as well and will return to producing when enough items have been consumed. Another test was to make sure that the buffer was only being edited by one thread at a time, this was done by printing the buffer in each of the respective functions (producer and consumer) during different parts of the flow and making sure nothing out of the ordinary was happening. Lastly we made sure that the producer was adding to the next open spot in the array and that the consumer was consuming from index zero. This was done by just printing the buffer after each action and manually checking. After testing all of these conditions, we were satisfied we came up with a good solution to the problem.}

\noindent\textbf{What did you learn?}

\indent{We learned more about working with pthreads than in the past. Conditional blocking with pthreads was a new skill learned which was extremely helpful. Also, although the concept of a producer-consumer threading system makes intuitive sense, neither Will or I had done this exact patter before so we learned a lot about the implementation of it by coding and looking at “The Little Book of Semaphores”.}

\section{Version Control Log}
\noindent Table from repo logs:
\begin{center}
    \begin{tabular}{ | l | l | l |}
    \hline
    Commit Link & Author & Commit Message \\ \hline
    \href{https://github.com/omalleya/cs444-concurrency-writeups/commit/7382ef98c7e6952423df0530a3c443b38f86211e}{7382ef98c7e6952423df0530a3c443b38f86211e} & omalleya & initial commit \\ \hline
    \href{https://github.com/omalleya/cs444-concurrency-writeups/commit/651ae7e67d3ab064dc3d4cd99fde1596640ae721}{651ae7e67d3ab064dc3d4cd99fde1596640ae721} & omalleya & adds .gitignore \\ \hline
    \href{https://github.com/omalleya/cs444-concurrency-writeups/commit/7ccb2584ad1152b69ca9e655bc237a2d4bc89066}{7ccb2584ad1152b69ca9e655bc237a2d4bc89066} & William Sims (wsims) & Added fixed LaTeX \\ \hline
    \href{https://github.com/omalleya/cs444-concurrency-writeups/commit/c70d2b6ec45f9af4cd0551a60a22cc42865816d3}{c70d2b6ec45f9af4cd0551a60a22cc42865816d3} & William Sims (wsims) & Removed old files \\ \hline
    \href{https://github.com/omalleya/cs444-concurrency-writeups/commit/2c83dec1d175599d2fad1a7c91064e7a414993e2}{2c83dec1d175599d2fad1a7c91064e7a414993e2} & omalleya & adds bash history for yocto \\ \hline
    \href{https://github.com/omalleya/cs444-concurrency-writeups/commit/bfa0c7df5b269563ff56092bfa4ec26ce3e9c1a3}{bfa0c7df5b269563ff56092bfa4ec26ce3e9c1a3} & William Sims (wsims) & For loop error \\ \hline
    \href{https://github.com/omalleya/cs444-concurrency-writeups/commit/9af0342269ec1c7d8015000c0c05a7f074d950d1}{9af0342269ec1c7d8015000c0c05a7f074d950d1} & William Sims (wsims) & Merge branch 'master' \\ \hline
    \href{https://github.com/omalleya/cs444-concurrency-writeups/commit/a10be60a3f9dafc59fd6e97cd0d6195d62e60c72}{a10be60a3f9dafc59fd6e97cd0d6195d62e60c72} & William Sims (wsims) & Added bash log \\ \hline
    \href{https://github.com/omalleya/cs444-concurrency-writeups/commit/e8918498f577329fd8043b82d910cd09b31e13e0}{e8918498f577329fd8043b82d910cd09b31e13e0} & omalleya & adds flag descriptions to .tex \\ \hline
    \href{https://github.com/omalleya/cs444-concurrency-writeups/commit/f8ae9944de0892b79e20e722506a09e800f9202c}{f8ae9944de0892b79e20e722506a09e800f9202c} & William Sims (wsims) & Created work log \\ \hline
    \end{tabular}
\end{center}

\section{Work Log}
\noindent What work was done and when:
\begin{description}
\item [10/2/17] Met at 2:00 in Johnson Hall and finished most of the Kernel portion of the assignment.
\item [10/3/17] Asked Kevin a question in class about using \$GDB and finished the Kernel portion of the assignment. 
\item [10/7/17] Met at 11:30 in Johnson. Planned out functions and pair programmed for the concurrency problem.
\item [10/7/17] Will created the LaTeX document and updated the makefile so it correctly compiles on the OSU server. 
\item [10/8/17] Aidan finished the concurrency problem remotely and created a GitHub repo for our concurrency problems. 
\item [10/8/17] Will added the log of commands to the LaTeX document. 
\item [10/9/17] Aidan added the flag definitions to the LaTex document. 
\end{description}

\end{document}
