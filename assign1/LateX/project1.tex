\documentclass[10pt,letterpaper,draftclsnofoot,onecolumn]{IEEEtran}

\usepackage{graphicx}                                        
\usepackage{amssymb}                                         
\usepackage{amsmath}                                         
\usepackage{amsthm}                                          

\usepackage{alltt}                                           
\usepackage{float}
\usepackage{color}
\usepackage{url}

\usepackage{balance}
\usepackage[TABBOTCAP, tight]{subfigure}
\usepackage{enumitem}
%\usepackage{pstricks, pst-node}

\usepackage{geometry}
\geometry{margin=0.75in}

%random comment

\newcommand{\cred}[1]{{\color{red}#1}}
\newcommand{\cblue}[1]{{\color{blue}#1}}

\newcommand{\toc}{\tableofcontents}

\usepackage{hyperref}

\def\name{Will Sims, Aidan O'Malley}

%pull in the necessary preamble matter for pygments output
%\input{pygments.tex}

%% The following metadata will show up in the PDF properties
\hypersetup{
   colorlinks = false,
   urlcolor = black,
   pdfauthor = {\name},
   pdfkeywords = {cs444},
   pdftitle = {CS 444 Project 1: Getting Acquainted},
   pdfsubject = {CS 444 Project 1},
   pdfpagemode = UseNone
}

\parindent = 0.0 in
\parskip = 0.1 in

\begin{document}

%input the pygmentized output of mt19937ar.c, using a (hopefully) unique name
%this file only exists at compile time. Feel free to change that.

\begin{titlepage}
\title{CS 444 Project 1: Getting Acquainted}
\author
{\IEEEauthorblockN{Will Sims, Aidan O'Malley\\
}
\IEEEauthorblockA{
CS 444\\
Fall 2017
}}
    \maketitle
    \vspace{2cm}
    \begin{abstract}
        \noindent Abstract goes here
    \end{abstract}

\end{titlepage}

\section{Log of Commands}
\noindent Log of commands used to perform the requested actions
\begin{enumerate}
\item \texttt{cd /scratch/fall2017}
\item \texttt{ls}
\item \texttt{mkdir 19}
\item \texttt{ls}
\item \texttt{19}
\item \texttt{git clone git://git.yoctoproject.org/linux-yocto-3.19}
\item \texttt{git status}
\item \texttt{git add .}
\item \texttt{git commit -m "initial commit"}
\item \texttt{git status}
\item \texttt{cd ..}
\item \texttt{cd fall2017}
\item \texttt{cd ..}
\item \texttt{./acl\_open ../fall2017/19 simsw}
\item \texttt{cd ..}
\item \texttt{cd fall2017}
\item \texttt{cd 19}
\item \texttt{cd linux-yocto-3.19/}
\item \texttt{git checkout v3.19.2}
\item \texttt{source ../../../opt/poky/1.8/environment-setup-i586-poky-linux }
\item \texttt{cp ../../../files/bzImage-qemux86.bin .}
\item \texttt{cp ../../../files/core-image-lsb-sdk-qemux86.ext4 .}
\item \texttt{make menuconfig}
\item \texttt{qemu-system-i386 -gdb tcp::5519 -S -nographic -kernel bzImage-qemux86.bin -drive file=core-image-lsb-sdk-qemux86.ext4,if=virtio -enable-kvm -net none -usb -localtime --no-reboot --append "root=/dev/vda rw console=ttyS0 debug"}
\item \texttt{\$GDB}
\item \texttt{target remote :5519}
\item \texttt{qemu-system-i386 -gdb tcp::5519 -S -nographic -kernel arch/x86/boot/bzImage -drive file=core-image-lsb-sdk-qemux86.ext4,if=virtio -enable-kvm -net none -usb -localtime --no-reboot --append "root=/dev/vda rw console=ttyS0 debug"}
\end{enumerate}

\section{Flag Definitions}
\noindent An explanation of each and every flag in the listed qemu command-line

\section{Concurrency Questions}
\noindent\textbf{What do you think the main point of this assignment is?}

\noindent\textbf{How did you personally approach the problem?}

\noindent\textbf{How did you ensure your solution was correct?}

\noindent\textbf{What did you learn?}

\section{Version Control Log}
\noindent Table from repo logs

\section{Work Log}
\noindent What was done when?

\end{document}
