\documentclass[10pt,letterpaper,draftclsnofoot,onecolumn]{IEEEtran}

\usepackage{graphicx}                                        
\usepackage{amssymb}                                         
\usepackage{amsmath}                                         
\usepackage{amsthm}                                          

\usepackage{alltt}                                           
\usepackage{float}
\usepackage{color}
\usepackage{url}

\usepackage{balance}
\usepackage[TABBOTCAP, tight]{subfigure}
\usepackage{enumitem}
%\usepackage{pstricks, pst-node}

\usepackage{geometry}
\geometry{margin=0.75in}

%random comment

\newcommand{\cred}[1]{{\color{red}#1}}
\newcommand{\cblue}[1]{{\color{blue}#1}}

\newcommand{\toc}{\tableofcontents}

\usepackage{hyperref}

\def\name{Will Sims, Aidan O'Malley}

%pull in the necessary preamble matter for pygments output
%\input{pygments.tex}

%% The following metadata will show up in the PDF properties
\hypersetup{
   colorlinks = false,
   urlcolor = black,
   pdfauthor = {\name},
   pdfkeywords = {cs444},
   pdftitle = {CS 444 Project 2: I/O Elevators},
   pdfsubject = {CS 444 Project 2},
   pdfpagemode = UseNone
}

\parindent = 0.0 in
\parskip = 0.1 in

\begin{document}

%input the pygmentized output of mt19937ar.c, using a (hopefully) unique name
%this file only exists at compile time. Feel free to change that.

\begin{titlepage}
\title{CS 444 Project 2: I/O Elevators}
\author
{\IEEEauthorblockN{Will Sims, Aidan O'Malley\\
}
\IEEEauthorblockA{
CS 444\\
Fall 2017
}}
    \maketitle
    \vspace{2cm}
    \begin{abstract}
        \noindent This document contains our write-up for Project 1 of Operating Systems II. Included is a list of log commands to setup our kernel, definitions of the flags used, solutions to the concurrency questions, a version control log, and work log. 
    \end{abstract}

\end{titlepage}

\section{Design for CLOOK Algorithm}
\noindent{}

\section{Kernel Questions}

\noindent\textbf{What do you think the main point of this assignment is?}

\indent{}

\noindent\textbf{How did you personally approach the problem?}

\indent{}

\noindent\textbf{How did you ensure your solution was correct?}

\indent{}

\noindent\textbf{What did you learn?}

\indent{}

\noindent\textbf{How should the TA evaluate your work? Provide detailed steps to prove correctness.}
\begin{description}
\item [Step 1] Blah.


\end{description}


\section{Version Control Log}
\noindent Table from repo logs:

\noindent \textbf{Note: } The concurrency portion of the assignment was done using paired programming without a central repo so a few commits don't have links. They do, however, accurately reflect how we completed that portion of the assignment.
\begin{center}
    \begin{tabular}{ | p{8cm} | p{3cm} | p{6cm} |}
    \hline
    Commit Link & Author & Commit Message \\ \hline
    \end{tabular}
\end{center}

\section{Work Log}
\noindent What work was done and when:
\begin{description}
\item [10/23/17] Met at Allan Bros for 2 hours and created skeleton for the dining philosophers problem.


\end{description}

\bibliography{project1}
\bibliographystyle{IEEEtran}

\end{document}
