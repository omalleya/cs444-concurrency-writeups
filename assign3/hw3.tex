\documentclass[10pt,letterpaper,draftclsnofoot,onecolumn]{IEEEtran}

\usepackage{graphicx}                                        
\usepackage{amssymb}                                         
\usepackage{amsmath}                                         
\usepackage{amsthm}                                          

\usepackage{alltt}                                           
\usepackage{float}
\usepackage{color}
\usepackage{url}

\usepackage{balance}
\usepackage[TABBOTCAP, tight]{subfigure}
\usepackage{enumitem}
%\usepackage{pstricks, pst-node}

\usepackage{geometry}
\geometry{margin=0.75in}

%random comment

\newcommand{\cred}[1]{{\color{red}#1}}
\newcommand{\cblue}[1]{{\color{blue}#1}}

\newcommand{\toc}{\tableofcontents}

\usepackage{hyperref}

\def\name{Will Sims, Aidan O'Malley}

%pull in the necessary preamble matter for pygments output
%\input{pygments.tex}

%% The following metadata will show up in the PDF properties
\hypersetup{
   colorlinks = false,
   urlcolor = black,
   pdfauthor = {\name},
   pdfkeywords = {cs444},
   pdftitle = {CS 444 Project 3: Encrypted Block Device},
   pdfsubject = {CS 444 Project 3},
   pdfpagemode = UseNone
}

\parindent = 0.0 in
\parskip = 0.1 in

\begin{document}

%input the pygmentized output of mt19937ar.c, using a (hopefully) unique name
%this file only exists at compile time. Feel free to change that.

\begin{titlepage}
\title{CS 444 Project 3: Encrypted Block Device}
\author
{\IEEEauthorblockN{Will Sims, Aidan O'Malley\\
}
\IEEEauthorblockA{
CS 444\\
Fall 2017
}}
    \maketitle
    \vspace{2cm}
    \begin{abstract}
        \noindent 
    \end{abstract}

\end{titlepage}

\section{Design for Encrypted Block Device}
\noindent{}

\section{Version Control Log}
\noindent Table from repo logs:

\begin{center}
    \begin{tabular}{ | p{8cm} | p{3cm} | p{6cm} |}
    \hline
    Commit Link & Author & Commit Message \\ \hline
    \href{https://github.com/omalleya/cs444-concurrency-writeups/commit/3f942907cd1aef64215014b8296081002eeedc8e}{3f942907cd1aef64215014b8296081002eeedc8e} & William Sims & Added to latex and created IO python script \\ \hline
    52703a0cbda4da1a56f5d784e7dc5acad19a40b0 & omalleya & reinitialize \\ \hline
    67e8554c6e400cb582486fe8c9141b15576443b0 & omalleya & creates simple block driver file in drivers/blocks/ and creates file skeleton \\ \hline
    a310be6ef0dd5e39fc01c53fa17edff8260f0210 & William Sims & adds simple block driver structure without encryption \\ \hline

    \end{tabular}
\end{center}

\section{Work Log}
\noindent What work was done and when:
\begin{description}
\item [11/11/17] Met at Johnson Hall at 2:00

\end{description}

\section{Questions}

\noindent\textbf{What do you think the main point of this assignment is?}

\indent{}

\noindent\textbf{How did you personally approach the problem?}

\indent{}

\noindent\textbf{How did you ensure your solution was correct?}

\indent{}

\noindent\textbf{What did you learn?}

\indent{}

\noindent\textbf{How should the TA evaluate your work? Provide detailed steps to prove correctness.}
\begin{description}
\item \texttt{[Step: 1] START FROM CLEAN linux-yocto v3.19.2}
\item \texttt{[Step: 2] git apply assign3.patch}
\item \texttt{[Step: 3] source /scratch/opt/poky/1.8/environment-setup-i586-poky-linux}
\item \texttt{[Step: 4] cp /scratch/files/core-image-lsb-sdk-qemux86.ext4 *top of linux tree* \newline //copy driver file if not inside already}
\item \texttt{[Step: 5] cp /scratch/files/config-3.19.2-yocto-standard *top of linux tree* \newline//copy config if not inside already}
\item \texttt{[Step: 6] make menuconfig}
\item \texttt{[Step: 7] General Setup}
\item \texttt{[Step: 8] Change local version name to “-group-19”}
\item \texttt{[Step: 9] Save and Exit}
\item \texttt{[Step: 10] make -j4 all \newline //build kernel \newline //select clook scheduler here or after starting VM run below command in VM}
\item \texttt{[Step: 11] echo clook > /sys/block/hda/queue/scheduler}
\item \texttt{[Step: 12] command to start VM: \newline qemu-system-i386 -gdb tcp::5519 -S -nographic -kernel arch/x86/boot/bzImage -drive file=core-image-lsb-sdk-qemux86.ext4 -enable-kvm -net none -usb -localtime --no-reboot --append "root=/dev/hda rw console=ttyS0 debug"}
\item \texttt{[Step: 13] \$GDB \newline //From another terminal}
\item \texttt{[Step: 14] target remote :5519}
\item \texttt{[Step: 15] continue}
\item \texttt{[Step: 16] Login to VM as root}
\item \texttt{[Step: 17] dd if=/dev/urandom of=./clook\_test bs=1024 count=8192 \newline //This will write random data to the disk and you should see the correct queue from the CLOOK output.} 
\item \texttt{[Step: 18] run command “dmesg” if logging to console is not sufficient to view the CLOOK operations}

\end{description}

\bibliography{hw3}
\bibliographystyle{IEEEtran}

\end{document}
