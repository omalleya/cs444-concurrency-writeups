\documentclass[10pt,letterpaper,draftclsnofoot,onecolumn]{IEEEtran}

\usepackage{graphicx}                                        
\usepackage{amssymb}                                         
\usepackage{amsmath}                                         
\usepackage{amsthm}                                          

\usepackage{alltt}                                           
\usepackage{float}
\usepackage{color}
\usepackage{url}

\usepackage{balance}
\usepackage[TABBOTCAP, tight]{subfigure}
\usepackage{enumitem}
%\usepackage{pstricks, pst-node}

\usepackage{geometry}
\geometry{margin=0.75in}

%random comment

\newcommand{\cred}[1]{{\color{red}#1}}
\newcommand{\cblue}[1]{{\color{blue}#1}}

\newcommand{\toc}{\tableofcontents}

\usepackage{hyperref}

\def\name{Will Sims, Aidan O'Malley}

%pull in the necessary preamble matter for pygments output
%\usepackage{fancyvrb}
\usepackage{color}
\usepackage[latin1]{inputenc}


\makeatletter
\def\PY@reset{\let\PY@it=\relax \let\PY@bf=\relax%
    \let\PY@ul=\relax \let\PY@tc=\relax%
    \let\PY@bc=\relax \let\PY@ff=\relax}
\def\PY@tok#1{\csname PY@tok@#1\endcsname}
\def\PY@toks#1+{\ifx\relax#1\empty\else%
    \PY@tok{#1}\expandafter\PY@toks\fi}
\def\PY@do#1{\PY@bc{\PY@tc{\PY@ul{%
    \PY@it{\PY@bf{\PY@ff{#1}}}}}}}
\def\PY#1#2{\PY@reset\PY@toks#1+\relax+\PY@do{#2}}

\expandafter\def\csname PY@tok@gd\endcsname{\def\PY@tc##1{\textcolor[rgb]{0.63,0.00,0.00}{##1}}}
\expandafter\def\csname PY@tok@gu\endcsname{\let\PY@bf=\textbf\def\PY@tc##1{\textcolor[rgb]{0.50,0.00,0.50}{##1}}}
\expandafter\def\csname PY@tok@gt\endcsname{\def\PY@tc##1{\textcolor[rgb]{0.00,0.25,0.82}{##1}}}
\expandafter\def\csname PY@tok@gs\endcsname{\let\PY@bf=\textbf}
\expandafter\def\csname PY@tok@gr\endcsname{\def\PY@tc##1{\textcolor[rgb]{1.00,0.00,0.00}{##1}}}
\expandafter\def\csname PY@tok@cm\endcsname{\let\PY@it=\textit\def\PY@tc##1{\textcolor[rgb]{0.25,0.50,0.50}{##1}}}
\expandafter\def\csname PY@tok@vg\endcsname{\def\PY@tc##1{\textcolor[rgb]{0.10,0.09,0.49}{##1}}}
\expandafter\def\csname PY@tok@m\endcsname{\def\PY@tc##1{\textcolor[rgb]{0.40,0.40,0.40}{##1}}}
\expandafter\def\csname PY@tok@mh\endcsname{\def\PY@tc##1{\textcolor[rgb]{0.40,0.40,0.40}{##1}}}
\expandafter\def\csname PY@tok@go\endcsname{\def\PY@tc##1{\textcolor[rgb]{0.50,0.50,0.50}{##1}}}
\expandafter\def\csname PY@tok@ge\endcsname{\let\PY@it=\textit}
\expandafter\def\csname PY@tok@vc\endcsname{\def\PY@tc##1{\textcolor[rgb]{0.10,0.09,0.49}{##1}}}
\expandafter\def\csname PY@tok@il\endcsname{\def\PY@tc##1{\textcolor[rgb]{0.40,0.40,0.40}{##1}}}
\expandafter\def\csname PY@tok@cs\endcsname{\let\PY@it=\textit\def\PY@tc##1{\textcolor[rgb]{0.25,0.50,0.50}{##1}}}
\expandafter\def\csname PY@tok@cp\endcsname{\def\PY@tc##1{\textcolor[rgb]{0.74,0.48,0.00}{##1}}}
\expandafter\def\csname PY@tok@gi\endcsname{\def\PY@tc##1{\textcolor[rgb]{0.00,0.63,0.00}{##1}}}
\expandafter\def\csname PY@tok@gh\endcsname{\let\PY@bf=\textbf\def\PY@tc##1{\textcolor[rgb]{0.00,0.00,0.50}{##1}}}
\expandafter\def\csname PY@tok@ni\endcsname{\let\PY@bf=\textbf\def\PY@tc##1{\textcolor[rgb]{0.60,0.60,0.60}{##1}}}
\expandafter\def\csname PY@tok@nl\endcsname{\def\PY@tc##1{\textcolor[rgb]{0.63,0.63,0.00}{##1}}}
\expandafter\def\csname PY@tok@nn\endcsname{\let\PY@bf=\textbf\def\PY@tc##1{\textcolor[rgb]{0.00,0.00,1.00}{##1}}}
\expandafter\def\csname PY@tok@no\endcsname{\def\PY@tc##1{\textcolor[rgb]{0.53,0.00,0.00}{##1}}}
\expandafter\def\csname PY@tok@na\endcsname{\def\PY@tc##1{\textcolor[rgb]{0.49,0.56,0.16}{##1}}}
\expandafter\def\csname PY@tok@nb\endcsname{\def\PY@tc##1{\textcolor[rgb]{0.00,0.50,0.00}{##1}}}
\expandafter\def\csname PY@tok@nc\endcsname{\let\PY@bf=\textbf\def\PY@tc##1{\textcolor[rgb]{0.00,0.00,1.00}{##1}}}
\expandafter\def\csname PY@tok@nd\endcsname{\def\PY@tc##1{\textcolor[rgb]{0.67,0.13,1.00}{##1}}}
\expandafter\def\csname PY@tok@ne\endcsname{\let\PY@bf=\textbf\def\PY@tc##1{\textcolor[rgb]{0.82,0.25,0.23}{##1}}}
\expandafter\def\csname PY@tok@nf\endcsname{\def\PY@tc##1{\textcolor[rgb]{0.00,0.00,1.00}{##1}}}
\expandafter\def\csname PY@tok@si\endcsname{\let\PY@bf=\textbf\def\PY@tc##1{\textcolor[rgb]{0.73,0.40,0.53}{##1}}}
\expandafter\def\csname PY@tok@s2\endcsname{\def\PY@tc##1{\textcolor[rgb]{0.73,0.13,0.13}{##1}}}
\expandafter\def\csname PY@tok@vi\endcsname{\def\PY@tc##1{\textcolor[rgb]{0.10,0.09,0.49}{##1}}}
\expandafter\def\csname PY@tok@nt\endcsname{\let\PY@bf=\textbf\def\PY@tc##1{\textcolor[rgb]{0.00,0.50,0.00}{##1}}}
\expandafter\def\csname PY@tok@nv\endcsname{\def\PY@tc##1{\textcolor[rgb]{0.10,0.09,0.49}{##1}}}
\expandafter\def\csname PY@tok@s1\endcsname{\def\PY@tc##1{\textcolor[rgb]{0.73,0.13,0.13}{##1}}}
\expandafter\def\csname PY@tok@sh\endcsname{\def\PY@tc##1{\textcolor[rgb]{0.73,0.13,0.13}{##1}}}
\expandafter\def\csname PY@tok@sc\endcsname{\def\PY@tc##1{\textcolor[rgb]{0.73,0.13,0.13}{##1}}}
\expandafter\def\csname PY@tok@sx\endcsname{\def\PY@tc##1{\textcolor[rgb]{0.00,0.50,0.00}{##1}}}
\expandafter\def\csname PY@tok@bp\endcsname{\def\PY@tc##1{\textcolor[rgb]{0.00,0.50,0.00}{##1}}}
\expandafter\def\csname PY@tok@c1\endcsname{\let\PY@it=\textit\def\PY@tc##1{\textcolor[rgb]{0.25,0.50,0.50}{##1}}}
\expandafter\def\csname PY@tok@kc\endcsname{\let\PY@bf=\textbf\def\PY@tc##1{\textcolor[rgb]{0.00,0.50,0.00}{##1}}}
\expandafter\def\csname PY@tok@c\endcsname{\let\PY@it=\textit\def\PY@tc##1{\textcolor[rgb]{0.25,0.50,0.50}{##1}}}
\expandafter\def\csname PY@tok@mf\endcsname{\def\PY@tc##1{\textcolor[rgb]{0.40,0.40,0.40}{##1}}}
\expandafter\def\csname PY@tok@err\endcsname{\def\PY@bc##1{\setlength{\fboxsep}{0pt}\fcolorbox[rgb]{1.00,0.00,0.00}{1,1,1}{\strut ##1}}}
\expandafter\def\csname PY@tok@kd\endcsname{\let\PY@bf=\textbf\def\PY@tc##1{\textcolor[rgb]{0.00,0.50,0.00}{##1}}}
\expandafter\def\csname PY@tok@ss\endcsname{\def\PY@tc##1{\textcolor[rgb]{0.10,0.09,0.49}{##1}}}
\expandafter\def\csname PY@tok@sr\endcsname{\def\PY@tc##1{\textcolor[rgb]{0.73,0.40,0.53}{##1}}}
\expandafter\def\csname PY@tok@mo\endcsname{\def\PY@tc##1{\textcolor[rgb]{0.40,0.40,0.40}{##1}}}
\expandafter\def\csname PY@tok@kn\endcsname{\let\PY@bf=\textbf\def\PY@tc##1{\textcolor[rgb]{0.00,0.50,0.00}{##1}}}
\expandafter\def\csname PY@tok@mi\endcsname{\def\PY@tc##1{\textcolor[rgb]{0.40,0.40,0.40}{##1}}}
\expandafter\def\csname PY@tok@gp\endcsname{\let\PY@bf=\textbf\def\PY@tc##1{\textcolor[rgb]{0.00,0.00,0.50}{##1}}}
\expandafter\def\csname PY@tok@o\endcsname{\def\PY@tc##1{\textcolor[rgb]{0.40,0.40,0.40}{##1}}}
\expandafter\def\csname PY@tok@kr\endcsname{\let\PY@bf=\textbf\def\PY@tc##1{\textcolor[rgb]{0.00,0.50,0.00}{##1}}}
\expandafter\def\csname PY@tok@s\endcsname{\def\PY@tc##1{\textcolor[rgb]{0.73,0.13,0.13}{##1}}}
\expandafter\def\csname PY@tok@kp\endcsname{\def\PY@tc##1{\textcolor[rgb]{0.00,0.50,0.00}{##1}}}
\expandafter\def\csname PY@tok@w\endcsname{\def\PY@tc##1{\textcolor[rgb]{0.73,0.73,0.73}{##1}}}
\expandafter\def\csname PY@tok@kt\endcsname{\def\PY@tc##1{\textcolor[rgb]{0.69,0.00,0.25}{##1}}}
\expandafter\def\csname PY@tok@ow\endcsname{\let\PY@bf=\textbf\def\PY@tc##1{\textcolor[rgb]{0.67,0.13,1.00}{##1}}}
\expandafter\def\csname PY@tok@sb\endcsname{\def\PY@tc##1{\textcolor[rgb]{0.73,0.13,0.13}{##1}}}
\expandafter\def\csname PY@tok@k\endcsname{\let\PY@bf=\textbf\def\PY@tc##1{\textcolor[rgb]{0.00,0.50,0.00}{##1}}}
\expandafter\def\csname PY@tok@se\endcsname{\let\PY@bf=\textbf\def\PY@tc##1{\textcolor[rgb]{0.73,0.40,0.13}{##1}}}
\expandafter\def\csname PY@tok@sd\endcsname{\let\PY@it=\textit\def\PY@tc##1{\textcolor[rgb]{0.73,0.13,0.13}{##1}}}

\def\PYZbs{\char`\\}
\def\PYZus{\char`\_}
\def\PYZob{\char`\{}
\def\PYZcb{\char`\}}
\def\PYZca{\char`\^}
\def\PYZam{\char`\&}
\def\PYZlt{\char`\<}
\def\PYZgt{\char`\>}
\def\PYZsh{\char`\#}
\def\PYZpc{\char`\%}
\def\PYZdl{\char`\$}
\def\PYZti{\char`\~}
% for compatibility with earlier versions
\def\PYZat{@}
\def\PYZlb{[}
\def\PYZrb{]}
\makeatother


%% The following metadata will show up in the PDF properties
\hypersetup{
   colorlinks = false,
   urlcolor = black,
   pdfauthor = {\name},
   pdfkeywords = {cs444},
   pdftitle = {CS 444 Project 3: Encrypted Block Device},
   pdfsubject = {CS 444 Project 3},
   pdfpagemode = UseNone
}

\parindent = 0.0 in
\parskip = 0.1 in

\begin{document}

%input the pygmentized output of mt19937ar.c, using a (hopefully) unique name
%this file only exists at compile time. Feel free to change that.

\begin{titlepage}
\title{CS 444 Project 3: Encrypted Block Device}
\author
{\IEEEauthorblockN{Will Sims, Aidan O'Malley\\
}
\IEEEauthorblockA{
CS 444\\
Fall 2017
}}
    \maketitle
    \vspace{2cm}
    \begin{abstract}
        \noindent 
    \end{abstract}

\end{titlepage}

\section{Design for Encrypted Block Device}
\noindent{Plan goes here}

\section{Version Control Log}
\noindent Table from repo logs:

\begin{center}
    \begin{tabular}{ | p{8cm} | p{3cm} | p{6cm} |}
    \hline
    Commit Link & Author & Commit Message \\ \hline
    \href{https://github.com/omalleya/cs444-concurrency-writeups/commit/3f942907cd1aef64215014b8296081002eeedc8e}{3f942907cd1aef64215014b8296081002eeedc8e} & William Sims & Added to latex and created IO python script \\ \hline
    52703a0cbda4da1a56f5d784e7dc5acad19a40b0 & omalleya & reinitialize \\ \hline
    67e8554c6e400cb582486fe8c9141b15576443b0 & omalleya & creates simple block driver file in drivers/blocks/ and creates file skeleton \\ \hline
    a310be6ef0dd5e39fc01c53fa17edff8260f0210 & William Sims & adds simple block driver structure without encryption \\ \hline
    85f0afa78e7c4a8c2f6a1cba496af4659fd83868 & omalleya & adds cipher initialization with key set up as well \\ \hline
    a78a861183eb7fc0d0540e0c489e13cf4bfd2a6e & omalleya & encrypts and decrypts data in transfer function \\ \hline
    041df9cabd16e0db54f1e695b7646d35b1354c79 & omalleya & adds key as module parameter \\ \hline
    fc01bd1245f73de1126d03d3fb4a28f54263ecd1 & omalleya & removes key from device structure since it is a global variable \\ \hline
    07e987ab34910de4990e81b1eaf0456ee2066adf & William Sims & updated Kconfig and Makefile \\ \hline
    a7ca4e8de175540eb76f24bcbfbc076335940d9c & smaller changes to sbd.c to get module to work inside of VM \\ \hline
    \end{tabular}
\end{center}

\section{Work Log}
\noindent What work was done and when:
\begin{description}
\item [11/11/17] Met at Johnson Hall at 2:00 and worked until 6:00. Finished programming the embedded block device.
\item [11/11/17] Stared latex write-up, added git and work log. 
\item [11/12/17] Aidan figured out how to use scp to transfer the module to our virtual machine.
\item [11/13/17] Added responses to assignment questions.

\end{description}

\section{Questions}

\noindent\textbf{What do you think the main point of this assignment is?}

\indent{}

\noindent\textbf{How did you personally approach the problem?}

\indent{ Our plan was based largely off of the LWN.net implementation of a simple block driver.
We knew that this block driver would be an adequate start and if we were able to add encryption to where the driver performed reads and writes we would have an acceptable driver. }

Looking at the LWN.net implementation made it clear that we'd need transfer, request, and ioctl functions to perform device operations.
We also knew we'd need an initialization and an exit function for those parts of a device's lifecycle.

For the request function we decided it would make the most sense to check what kind of request we're looking at.
If it wasn't a command then we skip it and continue looping through the queue, otherwise we send it to the transfer function which actually handles the request.

In the transfer function the main functionality we decided on was to check if we're reading or writing data.
This was going to come as a parameter to the function.
We decided if we're writing data we want to loop through the intended section of data and encrypt a block at a time from the request's “buffer” and write that encrypted block to the device's data member.
If we are reading with the current request we want to loop through the intended section of data and decrypt a block at a time from the device's data and write to the request's buffer.

The ioctl function is used to set up the geometry of the device.

The initialization function was the only other slightly complex function.
We took most of the initialization from the LWN.net implementation such as setting up the device with a spin lock, allocating memory, getting the request queue, registering the device, and setting up the Gendisk structure.
We also added code to the initialization to set up the encryption and decryption of the disk.
For this we decided we needed to allocate an AES crypto cipher and set that to a member of the device's structure for use in the transfer function.
We also decided this is where we'd set the cipher key for the block cipher.

\noindent\textbf{How did you ensure your solution was correct?}

\indent{We insured our solution was correct by static analysis first.
We then built the device directly into qemu to debug and be sure that the device would actually run in the VM.
After insuring this we loaded the device in as a module into the VM with scp and tested module parameters.
The actual test commands we ran can be found in the section below titled "How should the TA evaluate your work." }

\noindent\textbf{What did you learn?}

\indent{}

\noindent\textbf{How should the TA evaluate your work? Provide detailed steps to prove correctness.}
\begin{description}
\item \texttt{[Step: 1] START FROM CLEAN linux-yocto v3.19.2}
\item \texttt{[Step: 2] git apply assign3.patch}
\item \texttt{[Step: 3] source /scratch/opt/poky/1.8/environment-setup-i586-poky-linux}
\item \texttt{[Step: 4] cp /scratch/files/core-image-lsb-sdk-qemux86.ext4 *top of linux tree* \newline //copy driver file if not inside already}
\item \texttt{[Step: 5] cp /scratch/files/config-3.19.2-yocto-standard *top of linux tree* \newline//copy config if not inside already}
\item \texttt{[Step: 6] make menuconfig}
\item \texttt{[Step: 7] General Setup}
\item \texttt{[Step: 8] Change local version name to “-group-19”}
\item \texttt{[Step: 9] Go back to first page of menuconfig and go to Device Drivers}
\item \texttt{[Step: 10] Go to block devices and set Encrypted Block Device as a module}
\item \texttt{[Step: 11] Save and Exit}
\item \texttt{[Step: 12] make -j4 all \newline //build kernel}
\item \texttt{[Step: 13] command to start VM: \newline qemu-system-i386 -gdb tcp::5519 -S -nographic -kernel arch/x86/boot/bzImage -drive file=core-image-lsb-sdk-qemux86.ext4 -enable-kvm -net nic -net user,hostfwd=tcp::8080-:22 -usb -localtime --no-reboot --append "root=/dev/hda rw console=ttyS0 debug"}
\item \texttt{[Step: 14] \$GDB \newline //From another terminal}
\item \texttt{[Step: 15] target remote :5519}
\item \texttt{[Step: 16] continue}
\item \texttt{[Step: 17] scp -P 8080 ./drivers/block/sbd.ko root@localhost:~ \newline //From another terminal}
\item \texttt{[Step: 18] Login to VM as root and test with the following}
\item \texttt{[Step: 19] insmod sbd.ko key="1234567899"} 
\item \texttt{[Step: 20] mkfs.ext2 /dev/sbd0}
\item \texttt{[Step: 21] mkdir test}
\item \texttt{[Step: 22] mount /dev/sbd0 ./test}
\item \texttt{[Step: 23] echo ‘encrypt this message’ > ./test/encrypt}
\item \texttt{[Step: 24] grep -a ‘encrypt’ /dev/sbd0}
\item \texttt{[Step: 25] cat ./test/encrypt}
\item \texttt{[Step: 26] umount ./test}
\item \texttt{[Step: 27] rmmod sbd.ko}

\end{description}

\bibliography{hw3}
\bibliographystyle{IEEEtran}

\end{document}
