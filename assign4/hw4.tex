\documentclass[10pt,letterpaper,draftclsnofoot,onecolumn]{IEEEtran}
    
\usepackage{graphicx}                                        
\usepackage{amssymb}                                         
\usepackage{amsmath}                                         
\usepackage{amsthm}                                          

\usepackage{alltt}                                           
\usepackage{float}
\usepackage{color}
\usepackage{url}

\usepackage{balance}
\usepackage[TABBOTCAP, tight]{subfigure}
\usepackage{enumitem}
%\usepackage{pstricks, pst-node}

\usepackage{listings}
% Taken from http://timmurphy.org/2014/01/27/displaying-code-in-latex-documents/ with some modifications
\lstset{
    frame=none,
    tabsize=4, % tab space width
    showstringspaces=false, % don't mark spaces in strings
    numbers=none, 
    commentstyle=\color{green}, % comment color
    keywordstyle=\color{blue}, % keyword color
    stringstyle=\color{red}, % string color
    basicstyle={\small\ttfamily}, % font style
    belowcaptionskip=1\baselineskip
}

\usepackage{geometry}
\geometry{margin=0.75in}

%random comment

\newcommand{\cred}[1]{{\color{red}#1}}
\newcommand{\cblue}[1]{{\color{blue}#1}}

\newcommand{\toc}{\tableofcontents}

\usepackage{hyperref}

\def\name{Will Sims, Aidan O'Malley}

%pull in the necessary preamble matter for pygments output
%\usepackage{fancyvrb}
\usepackage{color}
\usepackage[latin1]{inputenc}


\makeatletter
\def\PY@reset{\let\PY@it=\relax \let\PY@bf=\relax%
    \let\PY@ul=\relax \let\PY@tc=\relax%
    \let\PY@bc=\relax \let\PY@ff=\relax}
\def\PY@tok#1{\csname PY@tok@#1\endcsname}
\def\PY@toks#1+{\ifx\relax#1\empty\else%
    \PY@tok{#1}\expandafter\PY@toks\fi}
\def\PY@do#1{\PY@bc{\PY@tc{\PY@ul{%
    \PY@it{\PY@bf{\PY@ff{#1}}}}}}}
\def\PY#1#2{\PY@reset\PY@toks#1+\relax+\PY@do{#2}}

\expandafter\def\csname PY@tok@gd\endcsname{\def\PY@tc##1{\textcolor[rgb]{0.63,0.00,0.00}{##1}}}
\expandafter\def\csname PY@tok@gu\endcsname{\let\PY@bf=\textbf\def\PY@tc##1{\textcolor[rgb]{0.50,0.00,0.50}{##1}}}
\expandafter\def\csname PY@tok@gt\endcsname{\def\PY@tc##1{\textcolor[rgb]{0.00,0.25,0.82}{##1}}}
\expandafter\def\csname PY@tok@gs\endcsname{\let\PY@bf=\textbf}
\expandafter\def\csname PY@tok@gr\endcsname{\def\PY@tc##1{\textcolor[rgb]{1.00,0.00,0.00}{##1}}}
\expandafter\def\csname PY@tok@cm\endcsname{\let\PY@it=\textit\def\PY@tc##1{\textcolor[rgb]{0.25,0.50,0.50}{##1}}}
\expandafter\def\csname PY@tok@vg\endcsname{\def\PY@tc##1{\textcolor[rgb]{0.10,0.09,0.49}{##1}}}
\expandafter\def\csname PY@tok@m\endcsname{\def\PY@tc##1{\textcolor[rgb]{0.40,0.40,0.40}{##1}}}
\expandafter\def\csname PY@tok@mh\endcsname{\def\PY@tc##1{\textcolor[rgb]{0.40,0.40,0.40}{##1}}}
\expandafter\def\csname PY@tok@go\endcsname{\def\PY@tc##1{\textcolor[rgb]{0.50,0.50,0.50}{##1}}}
\expandafter\def\csname PY@tok@ge\endcsname{\let\PY@it=\textit}
\expandafter\def\csname PY@tok@vc\endcsname{\def\PY@tc##1{\textcolor[rgb]{0.10,0.09,0.49}{##1}}}
\expandafter\def\csname PY@tok@il\endcsname{\def\PY@tc##1{\textcolor[rgb]{0.40,0.40,0.40}{##1}}}
\expandafter\def\csname PY@tok@cs\endcsname{\let\PY@it=\textit\def\PY@tc##1{\textcolor[rgb]{0.25,0.50,0.50}{##1}}}
\expandafter\def\csname PY@tok@cp\endcsname{\def\PY@tc##1{\textcolor[rgb]{0.74,0.48,0.00}{##1}}}
\expandafter\def\csname PY@tok@gi\endcsname{\def\PY@tc##1{\textcolor[rgb]{0.00,0.63,0.00}{##1}}}
\expandafter\def\csname PY@tok@gh\endcsname{\let\PY@bf=\textbf\def\PY@tc##1{\textcolor[rgb]{0.00,0.00,0.50}{##1}}}
\expandafter\def\csname PY@tok@ni\endcsname{\let\PY@bf=\textbf\def\PY@tc##1{\textcolor[rgb]{0.60,0.60,0.60}{##1}}}
\expandafter\def\csname PY@tok@nl\endcsname{\def\PY@tc##1{\textcolor[rgb]{0.63,0.63,0.00}{##1}}}
\expandafter\def\csname PY@tok@nn\endcsname{\let\PY@bf=\textbf\def\PY@tc##1{\textcolor[rgb]{0.00,0.00,1.00}{##1}}}
\expandafter\def\csname PY@tok@no\endcsname{\def\PY@tc##1{\textcolor[rgb]{0.53,0.00,0.00}{##1}}}
\expandafter\def\csname PY@tok@na\endcsname{\def\PY@tc##1{\textcolor[rgb]{0.49,0.56,0.16}{##1}}}
\expandafter\def\csname PY@tok@nb\endcsname{\def\PY@tc##1{\textcolor[rgb]{0.00,0.50,0.00}{##1}}}
\expandafter\def\csname PY@tok@nc\endcsname{\let\PY@bf=\textbf\def\PY@tc##1{\textcolor[rgb]{0.00,0.00,1.00}{##1}}}
\expandafter\def\csname PY@tok@nd\endcsname{\def\PY@tc##1{\textcolor[rgb]{0.67,0.13,1.00}{##1}}}
\expandafter\def\csname PY@tok@ne\endcsname{\let\PY@bf=\textbf\def\PY@tc##1{\textcolor[rgb]{0.82,0.25,0.23}{##1}}}
\expandafter\def\csname PY@tok@nf\endcsname{\def\PY@tc##1{\textcolor[rgb]{0.00,0.00,1.00}{##1}}}
\expandafter\def\csname PY@tok@si\endcsname{\let\PY@bf=\textbf\def\PY@tc##1{\textcolor[rgb]{0.73,0.40,0.53}{##1}}}
\expandafter\def\csname PY@tok@s2\endcsname{\def\PY@tc##1{\textcolor[rgb]{0.73,0.13,0.13}{##1}}}
\expandafter\def\csname PY@tok@vi\endcsname{\def\PY@tc##1{\textcolor[rgb]{0.10,0.09,0.49}{##1}}}
\expandafter\def\csname PY@tok@nt\endcsname{\let\PY@bf=\textbf\def\PY@tc##1{\textcolor[rgb]{0.00,0.50,0.00}{##1}}}
\expandafter\def\csname PY@tok@nv\endcsname{\def\PY@tc##1{\textcolor[rgb]{0.10,0.09,0.49}{##1}}}
\expandafter\def\csname PY@tok@s1\endcsname{\def\PY@tc##1{\textcolor[rgb]{0.73,0.13,0.13}{##1}}}
\expandafter\def\csname PY@tok@sh\endcsname{\def\PY@tc##1{\textcolor[rgb]{0.73,0.13,0.13}{##1}}}
\expandafter\def\csname PY@tok@sc\endcsname{\def\PY@tc##1{\textcolor[rgb]{0.73,0.13,0.13}{##1}}}
\expandafter\def\csname PY@tok@sx\endcsname{\def\PY@tc##1{\textcolor[rgb]{0.00,0.50,0.00}{##1}}}
\expandafter\def\csname PY@tok@bp\endcsname{\def\PY@tc##1{\textcolor[rgb]{0.00,0.50,0.00}{##1}}}
\expandafter\def\csname PY@tok@c1\endcsname{\let\PY@it=\textit\def\PY@tc##1{\textcolor[rgb]{0.25,0.50,0.50}{##1}}}
\expandafter\def\csname PY@tok@kc\endcsname{\let\PY@bf=\textbf\def\PY@tc##1{\textcolor[rgb]{0.00,0.50,0.00}{##1}}}
\expandafter\def\csname PY@tok@c\endcsname{\let\PY@it=\textit\def\PY@tc##1{\textcolor[rgb]{0.25,0.50,0.50}{##1}}}
\expandafter\def\csname PY@tok@mf\endcsname{\def\PY@tc##1{\textcolor[rgb]{0.40,0.40,0.40}{##1}}}
\expandafter\def\csname PY@tok@err\endcsname{\def\PY@bc##1{\setlength{\fboxsep}{0pt}\fcolorbox[rgb]{1.00,0.00,0.00}{1,1,1}{\strut ##1}}}
\expandafter\def\csname PY@tok@kd\endcsname{\let\PY@bf=\textbf\def\PY@tc##1{\textcolor[rgb]{0.00,0.50,0.00}{##1}}}
\expandafter\def\csname PY@tok@ss\endcsname{\def\PY@tc##1{\textcolor[rgb]{0.10,0.09,0.49}{##1}}}
\expandafter\def\csname PY@tok@sr\endcsname{\def\PY@tc##1{\textcolor[rgb]{0.73,0.40,0.53}{##1}}}
\expandafter\def\csname PY@tok@mo\endcsname{\def\PY@tc##1{\textcolor[rgb]{0.40,0.40,0.40}{##1}}}
\expandafter\def\csname PY@tok@kn\endcsname{\let\PY@bf=\textbf\def\PY@tc##1{\textcolor[rgb]{0.00,0.50,0.00}{##1}}}
\expandafter\def\csname PY@tok@mi\endcsname{\def\PY@tc##1{\textcolor[rgb]{0.40,0.40,0.40}{##1}}}
\expandafter\def\csname PY@tok@gp\endcsname{\let\PY@bf=\textbf\def\PY@tc##1{\textcolor[rgb]{0.00,0.00,0.50}{##1}}}
\expandafter\def\csname PY@tok@o\endcsname{\def\PY@tc##1{\textcolor[rgb]{0.40,0.40,0.40}{##1}}}
\expandafter\def\csname PY@tok@kr\endcsname{\let\PY@bf=\textbf\def\PY@tc##1{\textcolor[rgb]{0.00,0.50,0.00}{##1}}}
\expandafter\def\csname PY@tok@s\endcsname{\def\PY@tc##1{\textcolor[rgb]{0.73,0.13,0.13}{##1}}}
\expandafter\def\csname PY@tok@kp\endcsname{\def\PY@tc##1{\textcolor[rgb]{0.00,0.50,0.00}{##1}}}
\expandafter\def\csname PY@tok@w\endcsname{\def\PY@tc##1{\textcolor[rgb]{0.73,0.73,0.73}{##1}}}
\expandafter\def\csname PY@tok@kt\endcsname{\def\PY@tc##1{\textcolor[rgb]{0.69,0.00,0.25}{##1}}}
\expandafter\def\csname PY@tok@ow\endcsname{\let\PY@bf=\textbf\def\PY@tc##1{\textcolor[rgb]{0.67,0.13,1.00}{##1}}}
\expandafter\def\csname PY@tok@sb\endcsname{\def\PY@tc##1{\textcolor[rgb]{0.73,0.13,0.13}{##1}}}
\expandafter\def\csname PY@tok@k\endcsname{\let\PY@bf=\textbf\def\PY@tc##1{\textcolor[rgb]{0.00,0.50,0.00}{##1}}}
\expandafter\def\csname PY@tok@se\endcsname{\let\PY@bf=\textbf\def\PY@tc##1{\textcolor[rgb]{0.73,0.40,0.13}{##1}}}
\expandafter\def\csname PY@tok@sd\endcsname{\let\PY@it=\textit\def\PY@tc##1{\textcolor[rgb]{0.73,0.13,0.13}{##1}}}

\def\PYZbs{\char`\\}
\def\PYZus{\char`\_}
\def\PYZob{\char`\{}
\def\PYZcb{\char`\}}
\def\PYZca{\char`\^}
\def\PYZam{\char`\&}
\def\PYZlt{\char`\<}
\def\PYZgt{\char`\>}
\def\PYZsh{\char`\#}
\def\PYZpc{\char`\%}
\def\PYZdl{\char`\$}
\def\PYZti{\char`\~}
% for compatibility with earlier versions
\def\PYZat{@}
\def\PYZlb{[}
\def\PYZrb{]}
\makeatother


%% The following metadata will show up in the PDF properties
\hypersetup{
    colorlinks = false,
    urlcolor = black,
    pdfauthor = {\name},
    pdfkeywords = {cs444},
    pdftitle = {CS 444 Project 4: 4},
    pdfsubject = {CS 444 Project 4},
    pdfpagemode = UseNone
}

\parindent = 0.0 in
\parskip = 0.1 in

\begin{document}

%input the pygmentized output of mt19937ar.c, using a (hopefully) unique name
%this file only exists at compile time. Feel free to change that.

\begin{titlepage}
\title{CS 444 Project 3: Encrypted Block Device}
\author
{\IEEEauthorblockN{Will Sims, Aidan O'Malley\\
}
\IEEEauthorblockA{
CS 444\\
Fall 2017
}}
    \maketitle
    \vspace{2cm}
    \begin{abstract}
        \noindent This document contains the write-up for Project 4 of Operating Systems II.
    \end{abstract}

\end{titlepage}

\section{Design for SLOB SLAB}
\noindent{The SLOB allocator is a small and efficient allocator that uses a first-fit algorithm for memory allocation.
Before we started programming we first looked up documentation to understand the slob implementation and best-fit algorithms. 
After viewing slob.c, we noticed that the slob\_page\_alloc() implements a first-fit algorithm to search for a page that has enough space for a new request.
We plan to modify slob\_page\_alloc() and used this pseudocode as a reference for our best-fit algorithm: 

\lstset{language=C++,caption={Best-fit algorithm pseudocode\cite{bestfit}.}}
\begin{lstlisting}
size(block) = n + size(header) 
Scan free list for smallest block with nWords >= size(block) 
if block not found 
    Failure (Garbage collection) 
else if free block nWords >= size(block) + threshold* 
    Split into a free block and an in-use block 
    Free block nWords = Free block nWords - size(block) 
    In-use block nWords = size(block) 
    Return pointer to in-use block 
else 
    Unlink block from free list 
    Return pointer to block 
\end{lstlisting}

Reference: https://www.cs.rit.edu/~ark/lectures/gc/03\_03\_03.html

We also plan to modify slob\_alloc() so it selects the page in order of best fit before allocating the specific blocks. 
}

\section{Version Control Log}
\noindent Table from repo logs:

\begin{center}
    \begin{tabular}{ | p{8cm} | p{3cm} | p{6cm} |}
    \hline
    Commit Link & Author & Commit Message \\ \hline
    52703a0cbda4da1a56f5d784e7dc5acad19a40b0 & omalleya & reinitialize \\ \hline
 
    \end{tabular}
\end{center}

\section{Work Log}
\noindent What work was done and when:
\begin{description}
\item [11/30/17] Met at Kelley at 9:30 to work on the assignment.

\end{description}

\section{Questions}

\noindent\textbf{What do you think the main point of this assignment is?}

\indent{
}

\noindent\textbf{How did you personally approach the problem?}

\indent{

}

\noindent\textbf{How did you ensure your solution was correct?}

\indent{

}

\noindent\textbf{What did you learn?}

\indent{

}

\noindent\textbf{How should the TA evaluate your work? Provide detailed steps to prove correctness.}
\begin{description}
\item \texttt{[Step: 1] START FROM CLEAN linux-yocto v3.19.2}
\item \texttt{[Step: 2] git apply assign3.patch}
\item \texttt{[Step: 3] source /scratch/opt/poky/1.8/environment-setup-i586-poky-linux}
\item \texttt{[Step: 4] cp /scratch/files/core-image-lsb-sdk-qemux86.ext4 *top of linux tree* \newline //copy driver file if not inside already}
\item \texttt{[Step: 5] cp /scratch/files/config-3.19.2-yocto-standard *top of linux tree* \newline//copy config if not inside already}
\item \texttt{[Step: 6] make menuconfig}
\item \texttt{[Step: 7] General Setup}
\item \texttt{[Step: 8] Change local version name to “-group-19”}
\item \texttt{[Step: 9] Go back to first page of menuconfig and go to Device Drivers}
\item \texttt{[Step: 10] Go to block devices and set Encrypted Block Device as a module}
\item \texttt{[Step: 11] Save and Exit}
\item \texttt{[Step: 12] make -j4 all \newline //build kernel}
\item \texttt{[Step: 13] command to start VM: \newline qemu-system-i386 -gdb tcp::5519 -S -nographic -kernel arch/x86/boot/bzImage -drive file=core-image-lsb-sdk-qemux86.ext4 -enable-kvm -net nic -net user,hostfwd=tcp::8080-:22 -usb -localtime --no-reboot --append "root=/dev/hda rw console=ttyS0 debug"}
\item \texttt{[Step: 14] \$GDB \newline //From another terminal}
\item \texttt{[Step: 15] target remote :5519}
\item \texttt{[Step: 16] continue}
\item \texttt{[Step: 17] scp -P 8080 ./drivers/block/sbd.ko root@localhost:~ \newline //From another terminal}
\item \texttt{[Step: 18] Login to VM as root and test with the following}
\item \texttt{[Step: 19] insmod sbd.ko key="1234567899"} 
\item \texttt{[Step: 20] mkfs.ext2 /dev/sbd0}
\item \texttt{[Step: 21] mkdir test}
\item \texttt{[Step: 22] mount /dev/sbd0 ./test}
\item \texttt{[Step: 23] echo ‘encrypt this message’ > ./test/encrypt}
\item \texttt{[Step: 24] grep -a ‘encrypt’ /dev/sbd0}
\item \texttt{[Step: 25] cat ./test/encrypt}
\item \texttt{[Step: 26] umount ./test}
\item \texttt{[Step: 27] rmmod sbd.ko}

\end{description}

\bibliography{hw4}
\bibliographystyle{IEEEtran}

\end{document}
    
